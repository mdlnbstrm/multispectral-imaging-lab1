\subsection{Pixel size, magnification and resolution}
When we calculated length of the car in figure \ref{fig:Toy1} we obtained the result 10 cm as mentioned before. This was given by calculating the resolution using known length of the squares in the test chart standing in the background. When calculating the real length of the car by using a ruler we got the result 8.5 cm. This error comes from the fact that the car is placed in front of the chart and therefore the distance to the camera is shorter. When we placed an object closer to the camera, the resolution increased and we had more pixels per mm. 

\subsection{Grey scale image format and image information}

%discussion     ...we removed the last row because we wanted the matrices on the same format to be able to use subtraction. When we do the subtraction I1-I2 we will get the frames of the objects in the image. This is because of when we subtract each pixel with it's neighbor we will only se black if the two have the same color, and if the two pixels have different values, like the edge of the car and the background next to the car, we will see the frame of our object when the pixels are going from lighter to darker.


%grey values blabla function mesh shows different depths depending on what colour the pixels have

\subsection{Image enhancement}
\subsection{Colour images}
\subsection{IR - Imaging}