\subsection{Pixel size, magnification and resolution}
When we calculated length of the car in figure \ref{fig:Toy1} we obtained the result 10 cm as mentioned before. This was given by calculating the resolution using known length of the squares in the test chart standing in the background. When calculating the real length of the car by using a ruler we got the result 8.5 cm. This error comes from the fact that the car is placed in front of the chart and therefore the distance to the camera is shorter. When we placed an object closer to the camera, the resolution increased and we had more pixels per mm. 

\subsection{Grey scale image format and image information}
The function \texttt{mesh} creates a 3D plot where Z determines the height. The colour of the pixels are determined by the height. In figure \ref{fig:meshCar} the height is determined by the colour of the original image. The function \texttt{surfl} is very similar but it displays both the lines connecting the defining points and the faces of the surface in colour as can be seen in figure \ref{fig:meshCar}.

When producing the figure \ref{fig:I2_I1} we first shifted every pixel in the first matrix one row and then removed the last row of the other matrix because we wanted the matrices on the same format to be able to use subtraction between I1 and I2. When we did this we got the frames of the objects in the image, this is because of when we subtract each pixel with it's neighbor we will only see black if the two have the same color, and if the two pixels have different values, like the edge of the car and the background next to the car, we will see the frame of our object when the pixels are going from lighter to darker. In our experiment we shifted each row to simulate what kind of result we would be able to get if something had moved one pixel.

\subsection{Image enhancement}
In order to enhance the images in figure \ref{fig:img1org} and \ref{fig:img2org} we used histogram equalization and we can see in figure \ref{fig:img2new} and \ref{fig:img1new} that the images are much easier to examine with our eyes. 
\subsection{Colour images}

\subsection{IR - Imaging}