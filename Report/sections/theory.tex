\subsection{Magnification and resolution}
Resolution is the ability for an imaging system to distinguish object detail, which often is expressed in line-pairs per millimeter (lp/mm) also known as frequency. A digital image has it's dimension in pixels and the resolution of digital images is the dimension of an image in pixels. The ratio between image and object size is known as the magnification for an optical system. To measure the resolution one can use different methods such as Rayleigh's criterion, modulation transfer function (MTF) and point spread function (SPF). 

% Detta är för avskrivet behöver omformuleras.
When using Rayleigh's criterion one find the distance between the images of two points sources as they are resolved. A better way to measure the resolution is to determine the MTF of the system. 
% 
MTF is the ratio between input and output modulation/contrast, defined as \[ m = \frac{I_{max}-I_{min}}{I_{max}+I_{min}}. \] The contrast decreases when the spatial frequency of the lines increases. We estimate the resolution to be lines/mm where the MTF value is 0.5.

\subsection{Grey scale images, image format and image information}
Every pixel stored in the matrix as unsigned integer, which means that we can encounter some problems when performing mathematical operations, eg 150+150=255 or 10-33=0. For this reason we temporarily convert the matrix using the formula I=double(I)/255. 

When subtracting two images one subtracts each pixel in the first matrix with the corresponding pixel in the other matrix. %http://homepages.inf.ed.ac.uk/rbf/HIPR2/pixsub.htm
This means that if we subtract to identical images we will only get a black image since the intensity in each pixel will be 0, this also means that we could detect movement in the picture since some pixels will be displaced in the second image and the result will no longer be 0.
\subsection{Image enhancement}

\subsection{Colour images}

% Kom på något bättre till rubrik

\subsection{IR - Imaging}